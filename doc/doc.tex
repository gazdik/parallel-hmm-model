\documentclass[11pt,a4paper]{article}
\usepackage[slovak]{babel}
\usepackage[utf8]{inputenc}
\usepackage{times}
\usepackage{url}
\usepackage[textwidth=15.2cm,textheight=23cm]{geometry}
\usepackage{xcolor}
\usepackage{algorithm}
\usepackage{algpseudocode}
\usepackage{amsfonts}
\usepackage{amsmath}
\usepackage{calc}
\usepackage{enumitem}

\usepackage{tikz}
\usetikzlibrary{shapes.geometric, arrows}
\tikzstyle{start} = [circle, minimum width=0.5cm, minimum height=0.5cm, text centered, draw=black, fill=black]
\tikzstyle{stop} = [draw, double=white, circle, inner sep=1pt, minimum width=0.5cm, minimum height=0.5cm, draw=black]
\tikzstyle{io} = [trapezium, trapezium left angle=70, trapezium right angle=110, minimum width=3cm, minimum height=0.8cm, text centered, draw=black]
\tikzstyle{process} = [rectangle, minimum width=3cm, minimum height=0.8cm, text centered, draw=black]
\tikzstyle{decision} = [diamond, minimum width=3cm, minimum height=0.8cm, text centered, draw=black]
\tikzstyle{arrow} = [thick,->,>=stealth]

\usepackage{pgfplots}

\newcommand{\V}[1]{\mathit{#1}}
\newcommand{\K}[1]{\mathrm{#1}}
\DeclareMathOperator{\argmax}{arg\,max}

\usepackage{graphicx}

%\usepackage{fancyvrb}
%\DefineVerbatimEnvironment{verbatim}{Verbatim}{}

\usepackage[bf]{caption}

\usepackage[hyperindex,
  plainpages=false,
  pdftex,
  colorlinks,
  pdfborder={0 0 0},
  pdfpagelabels]{hyperref}

\pdfcompresslevel=9

\newcommand{\myincludegraphics}[4]{
  \begin{figure}[!h]
  \centering
  \includegraphics[#1]{#2}
  \caption{#3.} \label{#4}
  \end{figure}
}

% titulní stránka a obsah
\newcommand{\titlepageandcontents}{
  \input{titlepage.tex}

  \pagestyle{plain}
  \pagenumbering{roman}
  \setcounter{page}{1}
  %\tableofcontents

  \newpage
  \pagestyle{plain}
  \pagenumbering{arabic}
  \setcounter{page}{1}
}

\def\uv#1{\iflanguage{english}{``#1''}%
                              {\quotedblbase #1\textquotedblleft}}%

% vim:set ft=tex expandtab enc=utf8:


\begin{document}
\titlepageandcontents

%---------------------------------------------------------------------------
\section{Zadanie}

Predveďte praktickú úlohu riešenú pomocou Hidden Markov Model (HMM):
\begin{itemize}
\item zvoľte si úlohu, ktorú budete riešiť pomocou HMM
\item implementujte Forward algoritmus na CPU a GPU
\item implementujte Viterbiho algoritmus na CPU a GPU
\item popíšte zpôsob paralelizácie implementovaných algoritmov a použitý HW.
\item porovnajte časy behov algoritmov implementovaných na CPU a GPU
\item analyzujte časové zložitosti implementácii algoritmov na CPU a GPU
\end{itemize}
%Zde napište informace k zadání (nejde jen o přepis toho, co je na webu;
%komentujte vaše vlastní zpřesnění zadání, zaměření, důrazy, pojetí %atd.). Text
%strukturujte, použijte odrážky, číslování$\ldots$

%Rozsah: cca 10 odrážek

%---------------------------------------------------------------------------
\section{Použité technologie}
\begin{itemize}
\item C++
\item OpenCL
\item CMake
\end{itemize}
%Zde vypište, jaké technologie vaše řešení používá – co potřebuje k běhu, co
%jste použili při tvorbě, atd. Text strukturujte, použijte odrážky,
%číslování$\ldots$

%Rozsah: cca 7 odrážek

%---------------------------------------------------------------------------
\section{Použité zdroje}
\begin{itemize}
\item A tutorial on hidden Markov models and selected applications in speech recognition \cite{rabiner}
\item Optimizing parallel reduction in CUDA \cite{cuda-reductions}
\item NVIDIA OpenCL examples \url{https://github.com/sschaetz/nvidia-opencl-examples}
\item Speech and Language Processing: Hidden Markov Models \cite{jurafsky_martin}
\item Calculation of sum of logarithms \cite{computing-log-sum}
\item OpenCL optimization case study: Simple reductions \cite{amd-reductions}
\item GMU cvičenia
\end{itemize}

%Zde vypište, které zdroje jste použili k tvorbě: hotový kód, hotová data
%(obrázky, modely, $\ldots$), studijní materiály. Pokud vyplyne, že v projektu
%je použit kód nebo data, která nejsou uvedena tady, jedná se o závažný problém
%a projekt bude pravděpodobně hodnocen 0 body.

%Rozsah: potřebný počet odrážek

%---------------------------------------------------------------------------
\section{Najdôležitejšie dosiahnuté výsledky}
\begin{itemize}
\item niekoľko násobné zrýchlenie Forward a Viterbi algoritmu
\item zníženie časovej zložitosti algoritmov z $N^{2}*T$ na $log(N)*T$, kde $N$ je počet stavov HMM a $T$ je dĺžka reťazca pozorovania
\end{itemize}

\subsection{Zrýchlenie Forward algoritmu}

Pri Forward algoritme sa nám podarilo dosiahnuť až takmer 150 násobné zrýchlenie oproti CPU implementácii. Je to výrazne viac ako v prípade Viterbiho algoritmu, kde sme dosiahli maximálne 70 násobné zrýchlenie. Dôvodom by mohlo byť potreba sčítavania logaritmických hodnôt pri Forward algoritme, čo je časovo náročnejšie než len hľadanie maxima a preto paralelizácia tejto časti tak výrazne zvýši zrýchlenie. Detailnejšie výsledky je možné vidieť na obrázku \ref{forward-elapsed-time}. V časoch nie je zahrnutá doba potrebná na kopírovanie dát s maticami pravdepodobností na grafickú kartu, pretože aj pri bežnom použití sa tieto dáta kopírujú len pred prvým spustením a potom nad rôznymi pozorovaniami spúšťame Forward algoritmus opakovane. Len pre predstavu, kopírovanie matíc pravedpodobností pre 8192 stavov, ktoré zaberajú 270\,MB, trvá 227\,ms. Takže aj v prípade kopírovania pred každým spustením by bolo zrýchlenie takmer 140 násobné oproti CPU.

\begin{figure}[!ht]
  \centering
  \begin{tikzpicture}
    \begin{axis}[
      xbar,
      width=.8\textwidth,
      ylabel={Počet stavov},
      yticklabels = {512,1024,2048,4096,8192},
      ytick = {1,2,3,4,5},
      ytick pos = left,
      y dir = reverse,
      enlarge y limits={true, abs value = 0.5},
      y=1.1cm,
    	xlabel={Doba behu [s]},
      xtick pos = left,
    	enlarge x limits={upper, value=0.2},
      xmin=0,
      nodes near coords,
      reverse legend,
      every node near coord/.append style={
        /pgf/number format/fixed,
        /pgf/number format/set thousands separator={\,},
      },
      ylabel style={
        yshift = {1em},
      },
    ]
    \addplot
    	coordinates {(0.849,1) (4.2,2) (75.342,4) (18.621,3) (613.58,5)};
    \addplot
    	coordinates {(0.012,1) (0.0562,2) (0.145,3) (4.150,5) (0.465,4)};
    \legend{CPU,GPU}
    \end{axis}
  \end{tikzpicture}
  \caption{Doba behu Forward algoritmu pre pozorovania o dĺžke 20} \label{forward-elapsed-time}
\end{figure}

\newpage

\subsubsection{Zrýchlenie Viterbiho algoritmu}

GPU implementácia Viterbiho algoritmu dosahuje takmer 70 násobné zrýchlenie oproti oproti CPU implementácii. Výsledky zrýchlenia pre rôzne počty stavov zobrazuje obrázok \ref{viterbi-elapsed-time}.

\begin{figure}[!ht]
  \centering
  \begin{tikzpicture}
    \begin{axis}[
      xbar,
      width=.8\textwidth,
      ylabel={Počet stavov},
      yticklabels = {512,1024,2048,4096,8192},
      ytick = {1,2,3,4,5},
      ytick pos = left,
      y dir = reverse,
      enlarge y limits={true, abs value = 0.5},
      y=1.1cm,
    	xlabel={Doba behu [s]},
      xtick pos = left,
    	enlarge x limits={upper, value=0.2},
      xmin=0,
      nodes near coords,
      reverse legend,
      every node near coord/.append style={
        /pgf/number format/fixed,
        /pgf/number format/set thousands separator={\,},
      },
      ylabel style={
        yshift = {1em},
      }
    ]
    \addplot
    	coordinates {(0.273,1) (2.165,2) (9.498,3) (38.615,4) (271.53,5)};
    \addplot
    	coordinates {(0.057,1) (0.028,2) (0.0843,3) (0.3363,4) (3.931,5) };
    \legend{CPU,GPU}
    \end{axis}
  \end{tikzpicture}
  \caption{Doba behu Viterbiho algoritmu pre pozorovania o dĺžke 20} \label{viterbi-elapsed-time}
\end{figure}



% NAJLEPŠIE PRIDAŤ GRAFY
%Popište 3 věci, které jsou na vašem projektu nejlepší. Nejlépe %ukažte a
%komentujte obrázky, v nejhorším případě vypište textově.

%---------------------------------------------------------------------------

\section{Ovládanie vytvoreného programu}

Preklad aplikácie zabezpečuje nástroj CMake. Aplikácia sa spúšťa v príkazovom
riadku:

\begin{verbatim}
./hmm -s num_of_states -o num_of_outputs [-p platform] [-d device]
      [-l obs_length] [-O num_of_observations]
\end{verbatim}

\noindent kde jednotlivé parametre majú nasledovný význam:

\begin{description}
  \item[\texttt{-s}] počet stavov HMM,
  \item[\texttt{-o}] počet výstupov, ktoré HMM emituje,
  \item[\texttt{-p}] číslo OpenCL platformy, ktorá sa má použiť (z výpisu, programu), defaultne sa zvolí prvá dostupná,
  \item[\texttt{-d}] číslo OpenCL zariadenia, ktoré má byť použité (z výpisu programu), defaultne sa zvolí prvé dostupné,
  \item[\texttt{-l}] dĺžka jedného pozorovania, ktoré sa vyhodnocuje, defaultná hodnota je 30 symbolov,
  \item[\texttt{-O}] počet pozorovaní, defaultná hodnota je jedno pozorovanie.
\end{description}

\noindent Po spustení aplikácie s príslušnými parametrami dôjde k vytvoreniu HMM modelu s
definovaným počtom stavov a výstupov a jeho prechodové a emitujúce pravdepodobnosti sú náhodne inicalizované. Zároveň sa náhodne vygeneruje
požadovaný počet pozorovaní a nad daným modelom a pozorovaniami sa spustia jednotlivé algoritmy (na GPU aj CPU). Po skončení výpočtu príslušného algoritmu je vypísaná doba evaulácie daného algoritmu vrátane trvania jednotlivých fáz,
ktoré sú pre daný algoritmus typické.

%Stručně popište, jak se program ovládá (nejlépe odrážky rozdělené do
%kategorií). Pokud se ovládání odchyluje od zkratek a způsobů obvykle
%používaných v okýnkových nadstavbách operačních systémů, zdůvodněte, proč se
%tak děje.

%Rozsah: potřebný počet odrážek

%---------------------------------------------------------------------------
\section{Zvláštne použité znalosti}
%Uveďte informace, které byly potřeba nad rámec výuky probírané na %FIT.
%Vysvětlete je pomocí obrázků, schémat, vzorců apod.

\subsection{Skrytý Markov model}

Skrytý Markovov model (Hidden Markov Model - HMM) je štatistický Markov model,
ktorý modeluje systém za predpokladu, že ide o Markovov proces so skrytými
(nepozorovanými) stavmi.

Jednoduchý prípad Markovho modelu je Markovov reťazec, kde je stav systému viditeľný pozorovateľovi, takže pravdepodobnosť zmeny stavu je jediný parameter modelu. Naopak v skrytom Markovovom modeli nie je stav pozorovateľovi viditeľný,
ale len výstup, ktorý je na stave závislý. Každý stav má pravdepodobnostný vplyv na výstup systému. \cite{hmm-wiki}

Formálne je potom HMM definovaný nasledovnými komponentami \cite{jurafsky_martin}:
\begin{description}[leftmargin=!,labelwidth=12em]
  \item[$Q = q_1 q_2 \dots q_N$] konečná množina $N$ stavov,
  \item[$A = a_{1,1} a_{1,2} \dots a_{n,1} \dots a_{n,n}$] matica prechodových pravdepodobností $A$, kde $a_{i,j}$ reprezentuje pravdepodobnosť prechodu zo stavu $i$ do stavu $j$,
  \item[$O = o_1 o_2 \dots o_T$] postupnosť tvorená $T$ pozorovaniami, kde každé pozorovanie je nejaký symbol z abecedy $V = v_1, v_2, \dots, v_V$,
  \item[$B = b_i(o_t)$] postupnosť emitujúcich pravdepodobností, ktorá vyjadruje pravdepodobnosť vygenerovania pozorovania $o_t$ zo stavu $i$,
  \item[$\pi = \pi_1, \pi_2, \dots, \pi_N$]  počiatočné rozloženie prechodových pravdepodobností, kde $\pi_i$ je pravdepodobnosť, že Markovov reťazec začne v stave $i$,
  \item[$Q_A = \{q_x, q_y, \dots \}$] množina $Q_A \subseteq Q$ akceptujúcich stavov.
\end{description}

\bigskip

\noindent Pri práci s HMM potrebujemem riešiť 3 základné problémy \cite{rabiner}:
\begin{enumerate}
  \item K danému HMM $\lambda = (A, B)$ a postupnosti pozorovaní $O$ určiť pravdepodobnosť $P(O|\lambda))$.
  \item K danému HMM $\lambda = (A, B)$ a postupnosti pozorovaní $O$ nájsť najlepšiu postupnosť skrytých stavov $Q$.
  \item K danej postupnosti pozorovaní $O$ a množine stavov $Q$ HMM modelu určiť parametre $A$ a $B$.
\end{enumerate}

\noindent V rámci našej práce sa budeme zaoberať riešením a paralelizáciou prvých dvoch problémov, preto ich v následujúcich podkapitolách bližšie priblížime.

\subsection{Forward algoritmus}

Úlohou Forward algoritmu je určiť pravdepodobnosť výskytu konkrétnej postupnosti pozorovaní. Algoritmus výpočtu popisuje následujúci pseudokód:

\begin{algorithm}[!h]
  \floatname{algorithm}{Algoritmus}
  \caption{Forward algoritmus \cite{jurafsky_martin}} \label{alg-forward}
  \begin{algorithmic}[1]
    \renewcommand{\algorithmicrequire}{\textbf{Vstupy:}}
    \renewcommand{\algorithmicensure}{\textbf{Výstup:}}

    \Require probability matrix $A$, emission matrix $B$, observation $O$ of len $T$
    \Ensure $\V{likelihood}$

    \State create a probability matrix $\alpha[N, T]$
    \For{each state $s$ \textbf{from} $1$ \textbf{to} $N$} \Comment{Inicializačný krok}
      \State $\alpha[s,1] \gets \pi_{s} * b_s(o_1)$
    \EndFor
    \Statex
    \For{each time step $t$ \textbf{from} $2$ \textbf{to} $T$} \Comment{Rekurzívny krok}
      \For{each state $s$ \textbf{from} $1$ \textbf{to} $N$}
        \State $\alpha[s,t] \gets \sum_{s' = 1}^{N} \alpha[s', t - 1] * a_{s', s} * b_s(o_t)$
      \EndFor
    \EndFor
    \Statex
    \State $\V{likelihood} \gets \sum_{s = 1}^{N} \alpha[s,T]$ \Comment{Ukončovací krok}
  \end{algorithmic}
\end{algorithm}

Zložitosť tohto algoritmu je pri sekvenčnom vykonávaní $\mathcal{O}(N^2 \cdot T)$. V ideálnom prípade by sme s využitím paralelnej redukcie boli schopný dosiahnuť časovú zložitosť na úrovni  $\mathcal{O}(\log(N) \cdot T)$. To je však limitované počtom paralelných vlákien a pre veľký počet stavov nie sme schopný také zrýchlenie nikdy dosiahnuť. \cite{cuda-reductions}

Ďalšie urýchlenie výpočtu môžeme dosiahnuť počítaním s logaritmickými hodnotami pravdepodobností, vďaka čomu môžeme namiesto násobenia sčítať, čo je výrazne jednoduchšia operácia. Zároveň to rieši aj problém podtečenia, ktorý by sa musel riešiť normalizáciou výpočítaných hodnôt v každom kroku.

\newpage

\subsection{Viterbiho algoritmus}

Viterbiho algoritmus pre zadanú sekvenciu pozorovaní $O = o_1 o_2 \dots o_T$ hľadá najpravdepodobnejšiu sekvenciu stavov $Q = q_1 q_2\dots q_T$.

\begin{algorithm}[!h]
  \floatname{algorithm}{Algoritmus}
  \caption{Viterbiho algoritmus \cite{jurafsky_martin}} \label{alg-viterbi}
  \begin{algorithmic}[1]
    \renewcommand{\algorithmicrequire}{\textbf{Vstupy:}}
    \renewcommand{\algorithmicensure}{\textbf{Výstup:}}

    \Require probability matrix $A$, emission matrix $B$, observation $O$ of len $T$
    \Ensure viterbi path $\V{path}$

    \State create a path probability matrix $\V{v}[N, T]$
    \State create a backpointer matrix $\V{bt}[N, T]$

    \For{each state $s$ \textbf{from} $1$ \textbf{to} $N$} \Comment{Inicializačný krok}
      \State $\V{v}[s,1] \gets \pi_{s} * b_s(o_1)$
      \State $\V{bt}[s,1] \gets 0$
    \EndFor
    \Statex
    \For{each time step $t$ \textbf{from} $2$ \textbf{to} $T$} \Comment{Rekurzívny krok}
      \For{each state $s$ \textbf{from} $1$ \textbf{to} $N$}
        \State $\V{v}[s,t] \gets \max_{s' = 1}^{N} \alpha[s', t - 1] * a_{s', s} * b_s(o_t)$
        \State $\V{bt}[s,t] \gets \argmax_{s' = 1}^{N} \V{v}[s',t-1] * a_{s',s}$
      \EndFor
    \EndFor
    \Statex
    \State $\V{lastState} \gets \argmax_{s = 1}^{N} \V{v}[s,T]$ \Comment{Ukončovací krok}
    \State $\V{path} \gets$ the backtrace path by following backpointers $\V{bt}$ \Comment{Backtracing}
    \State \hskip\algorithmicindent to states back in time from $\V{lastState}$
  \end{algorithmic}
\end{algorithm}

\noindent Zložitosť je rovnaká ako v prípade Forward algoritmu a teda $\mathcal{O}(N^2 T)$.

%Rozsah: podle potřeby

%---------------------------------------------------------------------------
\section{Rozdelenie práce v tíme}

\begin{itemize}
\item Peter Gazdík: OpenCL réžia, CPU a GPU implementácia Viterbiho algoritmu, merania,
\item Michal Klčo: Forward algoritmus CPU a GPU, ladenie programu.
\end{itemize}

%Pokud to bude vhodné, použijte odrážky místo souvislých vět.

%Rozsah: co nejstručnější tak, aby bylo zřejmé, jak byla dělena práce a za co v
%projektu je kdo zodpovědný.

%---------------------------------------------------------------------------
\section{Čo bolo najpracnejšie}

Jedna z najťažších vecí na projekte bol návrh implementácie algoritmov tak, aby implementácie na GPU boli čo najrýchlejšie. To zahŕňalo redukciu časovej zložitosti a zlepšenie pamäťovej lokality. Veľa času sme tiež strávili odľadovaním chýb spojených s nepozornosťou pri voľaniach OpenCL API, ktoré vznikali aj napriek pomerne dôkladnej kontrole návratových hodnôt.
%Popište, co vám při řešení nejvíce komplikovalo život, s čím jste se museli
%potýkat, co zabralo čas.

%Rozsah: 5-10 řádků

%---------------------------------------------------------------------------
\section{Zkúsenosti získané riešením projektu}

Riešením projektu sme sa naučili, akým spôsobom HMM fungujú, ako sa počítajú parametre modelu a ďalšie s ním spojené hodnoty potrebné pre klasifikáciu. Naučili sme sa pracovať s OpenCL a implementovať algoritmy na GPU. Vyskúšali sme si tiež dekompozíciu zložitého problému a rozdelenie podúloh medzi jednotlivých členov týmu.
%^Popište, co jste se řešením projektu naučili. Zahrňte dovednosti obecně
%programátorské, věci z oblasti počítačové grafiky, ale i spolupráci v týmu,
%hospodaření s časem, atd.

%Rozsah: formulujte stručně, uchopte cca 3-5 věcí

%---------------------------------------------------------------------------
\section{Autoevaluácia}

%Ohodnoťte vaše řešení v jednotlivých kategoriích (0 – nic neuděláno,
%zoufalství, 100\% – dokonalost sama). Projekt, který ve finále obdrží plný
%počet bodů, může mít složky hodnocené i hodně nízko. Uvedení hodnot blízkých
%100\% ve všech nebo mnoha kategoriích může ukazovat na nepochopení problematiky
%nebo na snahu kamuflovat slabé stránky projektu. Bodově hodnocena bude i
%schopnost vnímat silné a slabé stránky svého řešení.

\paragraph{Technický návrh (80\%):}
% (analýza, dekompozice problému, volba
% vhodných prostředků, $\ldots$)
%Stručně (1-2 řádky) komentujte hodnocení.
Pri implementácii algoritmov sme sa držali striktne ich matematického popisu. Takto definované algoritmy sme najskôr implementovali na CPU a identifikovali sme časovo najkritickejšie časti, ktoré sme následne paralelizovali na GPU.

\paragraph{Programovanie (70\%):}
%(kvalita a čitelnost kódu, spolehlivost běhu,
%obecnost řešení, znovupoužitelnost, $\ldots$)
%Stručně (1-2 řádky) komentujte hodnocení.

\begin{itemize}
  \item Použitie OOP kvôli prehľadnosti a oddeleniu volaní OpenCL API od zvyšku aplikácie. Zapúzdrenie HMM a jednotlivých algoritmov, ktoré s ním pracujú.
  \item Využitie C++11 a moderných konštrukcií. (šablóny, \dots)
  \item Snaha o intuitívne názvy identifikátorov.
\end{itemize}

\paragraph{Vzhľad vytvoreného riešenia(30\%):}
%(uvěřitelnost zobrazení,
%estetická kvalita, vhled GUI, $\ldots$)
%Stručně (1-2 řádky) komentujte hodnocení.

Aplikácia je programovaná ako jednoduchý CLI program.

\paragraph{Využitie zdrojov (85\%):}

Pre vytvorenie programu bola využité kvalitne spracované zdroje popisujúce téoriu problematiky a existujúca implementácia poslúžila ako ukážka možného riešenia.
%(využití existujícího kódu a dat, využití
%literatury, $\ldots$)
%Stručně (1-2 řádky) komentujte hodnocení.

\paragraph{Hospodárenie s časom (50\%):}

Projekt sme nevypracovávali na poslednú chvíľu, ale pracovali sme na ňom priebežne počas semestra. Napriek tomu sme neimplementovali Baum-Welchov algoritmus, ktorý nie je dobre paralelizovateľný, čo dokazujú aj výsledky
iných prác.
% TODO citácia

%(rovnoměrné dotažení částí projektu,
%míra spěchu, chybějící části řešení, $\ldots$)
%Stručně (1-2 řádky) komentujte hodnocení.

\paragraph{Spolupráca v tíme (90\%):}

Každý člen týmu vypracoval to, čo po mu bolo po dohode pridelené. Komunikácia, zdieľanie zdrojov a priebežná kontrola členov bola podporená využívaním nástrojov ako git, Trello a pod.

%(komunikace, dodržování dohod, vzájemné
%polehnutí, rovnoměrnost, $\ldots$)
%Stručně (1-2 řádky) komentujte hodnocení.

\paragraph{Celkový dojem (85\%):}

Obťiažnosť projektu bola pre nás primeraná. Vďaka tomu, že nám bolo umožnené pracovať na nami zvolenej téme, bola práca pre náš tým zaujímavejšia. Za veľký prínos považujeme oboznámenie sa s OpenCL knižnicou a s paralelizáciou výpočtu. Ďalšou výhodou bolo, že sme si pripomenuli a prehĺbili teóriu HMM.

%(pracnost, získané dovednosti, užitečnost,
%volba zadání, cokoliv, $\ldots$)
%Stručně (5-10 řádků) komentujte hodnocení.

%---------------------------------------------------------------------------
\section{Doporučenie pre budúce zadávanie projektov}

Možnosť zvoliť si vlastnú tému a konzultovať zadanie bolo pre nás dôležitým spestrením projektu, pretože sme mali možnosť venovať sa veciam, ktoré nás viacej zaujímajú. Na organizácii by sme nič nemenili.

%Co vám vyhovovalo a co nevyhovovalo na organizaci projektů? Které prvky by měly
%být zachovány, zesíleny, potlačeny, eliminovány?

%---------------------------------------------------------------------------
% \section{Různé}

%Ještě něco by v dokumentaci mělo být? Napište to sem! Podle potřeby i založte
%novou kapitolu.

\bibliographystyle{plain}

\bibliography{reference}
% \addcontentsline{toc}{chapter}{Literatura}

\end{document}
% vim:set ft=tex expandtab enc=utf8:
