\documentclass[11pt,a4paper]{article}
\usepackage[slovak]{babel}
\usepackage[utf8]{inputenc}
\usepackage{times}
\usepackage{url}
\usepackage[textwidth=15.2cm,textheight=23cm]{geometry}
\usepackage{xcolor}
\usepackage{algorithm}
\usepackage{algpseudocode}
\usepackage{amsfonts}
\usepackage{amsmath}
\usepackage{calc}
\usepackage{enumitem}

\usepackage{tikz}
\usetikzlibrary{shapes.geometric, arrows}
\tikzstyle{start} = [circle, minimum width=0.5cm, minimum height=0.5cm, text centered, draw=black, fill=black]
\tikzstyle{stop} = [draw, double=white, circle, inner sep=1pt, minimum width=0.5cm, minimum height=0.5cm, draw=black]
\tikzstyle{io} = [trapezium, trapezium left angle=70, trapezium right angle=110, minimum width=3cm, minimum height=0.8cm, text centered, draw=black]
\tikzstyle{process} = [rectangle, minimum width=3cm, minimum height=0.8cm, text centered, draw=black]
\tikzstyle{decision} = [diamond, minimum width=3cm, minimum height=0.8cm, text centered, draw=black]
\tikzstyle{arrow} = [thick,->,>=stealth]

\usepackage{pgfplots}

\newcommand{\V}[1]{\mathit{#1}}
\newcommand{\K}[1]{\mathrm{#1}}
\DeclareMathOperator{\argmax}{arg\,max}

\usepackage{graphicx}

%\usepackage{fancyvrb}
%\DefineVerbatimEnvironment{verbatim}{Verbatim}{}

\usepackage[bf]{caption}

\usepackage[hyperindex,
  plainpages=false,
  pdftex,
  colorlinks,
  pdfborder={0 0 0},
  pdfpagelabels]{hyperref}

\pdfcompresslevel=9

\newcommand{\myincludegraphics}[4]{
  \begin{figure}[!h]
  \centering
  \includegraphics[#1]{#2}
  \caption{#3.} \label{#4}
  \end{figure}
}

% titulní stránka a obsah
\newcommand{\titlepageandcontents}{
  \input{titlepage.tex}

  \pagestyle{plain}
  \pagenumbering{roman}
  \setcounter{page}{1}
  %\tableofcontents

  \newpage
  \pagestyle{plain}
  \pagenumbering{arabic}
  \setcounter{page}{1}
}

\def\uv#1{\iflanguage{english}{``#1''}%
                              {\quotedblbase #1\textquotedblleft}}%

% vim:set ft=tex expandtab enc=utf8:


\begin{document}
\titlepageandcontents

%---------------------------------------------------------------------------
\section{Zadání}
Predveďte praktickú úlohu riešenú pomocou Hidden Markov Model (HMM):
\begin{itemize}
\item zvoľte si úlohu, ktorú budete riešiť pomocou HMM
\item implementujte \textit{forward} algoritmus na CPU a GPU
\item implementujte \textit{viterbi} algoritmus na CPU a GPU
\item popíšte zpôsob paralelizácie implementovaných algoritmov a použitý HW. 
\item porovnajte časy behov algoritmov implementovaných na CPU a GPU
\item analyzujte časové zložitosti implementácii algoritmov na CPU a GPU
\end{itemize}
%Zde napište informace k zadání (nejde jen o přepis toho, co je na webu;
%komentujte vaše vlastní zpřesnění zadání, zaměření, důrazy, pojetí %atd.). Text
%strukturujte, použijte odrážky, číslování$\ldots$

%Rozsah: cca 10 odrážek

%---------------------------------------------------------------------------
\section{Použité technologie}
\begin{itemize}
\item C++
\item OpenCL
\item CMake
\end{itemize}
%Zde vypište, jaké technologie vaše řešení používá – co potřebuje k běhu, co
%jste použili při tvorbě, atd. Text strukturujte, použijte odrážky,
%číslování$\ldots$

%Rozsah: cca 7 odrážek

%---------------------------------------------------------------------------
\section{Použité zdroje}
\begin{itemize}
\item RABINER, Lawrence R. A tutorial on hidden Markov models and selected applications in speech recognition. Proceedings of the IEEE, 1989, 77.2: 257-286.
\item NVIDIA OpenCL examples \url{https://github.com/sschaetz/nvidia-opencl-examples}
\item Calculation of sum of logarithms \url{https://hips.seas.harvard.edu/blog/2013/01/09/computing-log-sum-exp/}
\item Simple reduction \url{https://developer.amd.com/resources/articles-whitepapers/opencl-optimization-case-study-simple-reductions/}
\item GMU cvičenia
\end{itemize}

%Zde vypište, které zdroje jste použili k tvorbě: hotový kód, hotová data
%(obrázky, modely, $\ldots$), studijní materiály. Pokud vyplyne, že v projektu
%je použit kód nebo data, která nejsou uvedena tady, jedná se o závažný problém
%a projekt bude pravděpodobně hodnocen 0 body.

%Rozsah: potřebný počet odrážek

%---------------------------------------------------------------------------
\section{Nejdůležitější dosažené výsledky}
\begin{itemize}
\item niekoľko násobné zrýchlenie \textit{forward} a \textit{viterbi} algoritmu
\item zníženie časovej zložitosti algoritmov z $N^{2}*T$ na $log(N)*T$, kde $N$ je počet stavov HMM a $T$ je dĺžka reťazca pozorovania
\end{itemize}
% NAJLEPŠIE PRIDAŤ GRAFY
%Popište 3 věci, které jsou na vašem projektu nejlepší. Nejlépe %ukažte a
%komentujte obrázky, v nejhorším případě vypište textově.

%---------------------------------------------------------------------------
\section{Ovládání vytvořeného programu}
\begin{itemize}
\item Preklad: pomocou CMake
\item Beh: ./hmm $<$FIRST\_ARGUMENT$>$ $<$SECOND\_ARGUMENT$>$ $<$TRANSMISSION\_PROB\_FILE$>$ $<$EMIT\_PROB\_FILE$>$
\begin{description}
\item [FIRST\_ARGUMENT] description
\item [SECOND\_ARGUMENT] description
\item [TRANSMISION\_PROB\_FILE] - súbor s prechodovými pravdepodobnasťami pre každé dva stavy HMM
\item [EMIT\_PROB\_FILE] - súbor s emitujúcimi pravdepodobnosťami pre každý stav a každé pozorovanie HMM
\end{description}
\end{itemize}
%Stručně popište, jak se program ovládá (nejlépe odrážky rozdělené do
%kategorií). Pokud se ovládání odchyluje od zkratek a způsobů obvykle
%používaných v okýnkových nadstavbách operačních systémů, zdůvodněte, proč se
%tak děje.

%Rozsah: potřebný počet odrážek

%---------------------------------------------------------------------------
\section{Zvláštní použité znalosti}
\begin{itemize}
\item teória HMM
\end{itemize}
%Uveďte informace, které byly potřeba nad rámec výuky probírané na %FIT.
%Vysvětlete je pomocí obrázků, schémat, vzorců apod. 

%Rozsah: podle potřeby 

%---------------------------------------------------------------------------
\section{Rozdělení práce v týmu}

\begin{itemize}
\item Peter Gazdík: OpenCL réžia, Viterbi algoritmus CPU a GPU, merania
\item Michal Klčo: Forward algoritmus CPU a GPU, ladenie programu
\end{itemize}
%Pokud to bude vhodné, použijte odrážky místo souvislých vět.

%Rozsah: co nejstručnější tak, aby bylo zřejmé, jak byla dělena práce a za co v
%projektu je kdo zodpovědný.

%---------------------------------------------------------------------------
\section{Co bylo nejpracnější}
Jedna z najťažších vecí na projekte bol návrh implementácie algoritmov tak, aby implementácie na GPU boli čo najrýchlejšie. To zahŕňalo redukciu časovej zložitosti a zlepšenie pamäťovej lokality. Veľa čase sme strávili učením aplikačného rozhrania OpenCL.
%Popište, co vám při řešení nejvíce komplikovalo život, s čím jste se museli
%potýkat, co zabralo čas.

%Rozsah: 5-10 řádků

%---------------------------------------------------------------------------
\section{Zkušenosti získané řešením projektu}
Riešením projektu sme sa naučili, akým spôsobom HMM fungujú, ako sa počítajú parametre modelu a ďalšie s ním spojené hodnoty potrebné pre klasifikáciu. Naučili sme sa pracovať s OpenCL a implementovať algoritmy na GPU. Vyskúšali sme si tiež dekompozíciu zložitého problému a rozdelenie podúloh medzi jednotlivých členov týmu.
%^Popište, co jste se řešením projektu naučili. Zahrňte dovednosti obecně
%programátorské, věci z oblasti počítačové grafiky, ale i spolupráci v týmu,
%hospodaření s časem, atd.

%Rozsah: formulujte stručně, uchopte cca 3-5 věcí

%---------------------------------------------------------------------------
\section{Autoevaluace}

%Ohodnoťte vaše řešení v jednotlivých kategoriích (0 – nic neuděláno,
%zoufalství, 100\% – dokonalost sama). Projekt, který ve finále obdrží plný
%počet bodů, může mít složky hodnocené i hodně nízko. Uvedení hodnot blízkých
%100\% ve všech nebo mnoha kategoriích může ukazovat na nepochopení problematiky
%nebo na snahu kamuflovat slabé stránky projektu. Bodově hodnocena bude i
%schopnost vnímat silné a slabé stránky svého řešení.

\paragraph{Technický návrh (80\%):} 
%(analýza, dekompozice problému, volba
%vhodných prostředků, $\ldots$) 
%Stručně (1-2 řádky) komentujte hodnocení. 

\paragraph{Programování (70\%):} 
%(kvalita a čitelnost kódu, spolehlivost běhu,
%obecnost řešení, znovupoužitelnost, $\ldots$)
%Stručně (1-2 řádky) komentujte hodnocení. 

\paragraph{Vzhled vytvořeného řešení (30\%):}
Aplikácia je programovaná ako jednoduchý CLI program.
%(uvěřitelnost zobrazení,
%estetická kvalita, vhled GUI, $\ldots$)
%Stručně (1-2 řádky) komentujte hodnocení. 

\paragraph{Využití zdrojů (85\%):} 
Pre vytvorenie programu bola využité kvalitne spracované zdroje popisujúce téoriu problematiky a existujúca implementácia poslúžila ako ukážka možného riešenia.
%(využití existujícího kódu a dat, využití
%literatury, $\ldots$)
%Stručně (1-2 řádky) komentujte hodnocení. 

\paragraph{Hospodaření s časem (50\%):} 
Projekt sme nevypracovávali na poslednú chvíľu, ale nepracovali sme na ňom priebežne počas semestra.
%(rovnoměrné dotažení částí projektu,
%míra spěchu, chybějící části řešení, $\ldots$)
%Stručně (1-2 řádky) komentujte hodnocení. 

\paragraph{Spolupráce v týmu (90\%):}
Každý člen týmu vypracoval to, čo po mu bolo po dohode pridelené. Komunikácia, zdieľanie zdrojov a priebežná kontrola členov bola podporená využívaním nástrojov ako git, trello a pod.
%(komunikace, dodržování dohod, vzájemné
%polehnutí, rovnoměrnost, $\ldots$)
%Stručně (1-2 řádky) komentujte hodnocení. 

\paragraph{Celkový dojem (85\%):} 
Obťiažnosť projektu bola pre nás primeraná. Vďaka tomu, že nám bolo umožnené pracovať na nami zvolenej téme, bola práca pre náš tým zaujímavejšia. Za veľký prínos považujeme oboznámenie sa s OpenCL knižnicou a s paralelizáciou výpočtu. Ďalšou výhodou bolo, že sme si pripomenuli teóriu HMM.
%(pracnost, získané dovednosti, užitečnost,
%volba zadání, cokoliv, $\ldots$)
%Stručně (5-10 řádků) komentujte hodnocení. 

%---------------------------------------------------------------------------
\section{Doporučení pro budoucí zadávání projektů}
Možnosť zvoliť si vlastnú tému a konzultovať zadanie bolo pre nás dôležitým spestrením projektu, pretože sme mali možnosť venovať sa veciam, ktoré nás viacej zaujímali.
%Co vám vyhovovalo a co nevyhovovalo na organizaci projektů? Které prvky by měly
%být zachovány, zesíleny, potlačeny, eliminovány?

%---------------------------------------------------------------------------
\section{Různé}

%Ještě něco by v dokumentaci mělo být? Napište to sem! Podle potřeby i založte
%novou kapitolu.

\end{document}
% vim:set ft=tex expandtab enc=utf8:
